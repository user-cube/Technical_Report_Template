%!TEX root = ../Report.tex
\chapter{Introdução}
\label{chp:introduction}
Introdução
\section{Tópico}
1 TÓPICO
\subsection{Sub-tópico}
1 sub-tópico
\subsubsection{Sub-sub-tópico}
1 sub-sub-tópico

\section{Imagens e Tabelas}
\begin{figure}[!ht]
    \centering
    \includegraphics[width=0.5\textwidth]{images/ua.png}
    \caption{Exemplo de imagem}
    \label{fig:example}
\end{figure}
Os casos de usos consoante a sua prioridade podem sintetizados de acordo com a seguinte tabela:
\begin{table}[ht]
    \centering
        \begin{tabular}{p{.25\textwidth}p{.50\textwidth}p{.15\textwidth}}
            \hline
            \textbf{Coluna I} &	\textbf{Coluna II} &	\textbf{Coluna III} \\ 
            \hline
            Linha I & Linha I & Linha I \\
            \hline
        \end{tabular}
    \caption{Exemplo de uma tabela}
    \label{myTable}
\end{table}

\section{Enumerações}
Enumeração por tópicos
\begin{itemize}
    \item Elemento 1
          \SubItem{Sub elemento}
    \item Elemento 2
\end{itemize}
Enumeração por tópicos sem espaçamento:
\begin{itemize}[noitemsep]
    \item Elemento 1
          \SubItem{Sub elemento}
    \item Elemento 2
\end{itemize}
Enumeração:
\begin{enumerate}
    \item Elemento 1
    \item Elemento 2
\end{enumerate}
Enumeração sem espaçamento:
\begin{enumerate}[noitemsep]
    \item Elemento 1
    \item Elemento 2
\end{enumerate}
\newpage
\hfill\break